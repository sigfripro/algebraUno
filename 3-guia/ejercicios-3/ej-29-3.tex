\begin{enunciado}{\ejercicio}
  Sea $X = \set{1,2,\dots,20}$, y sea $R$ la relación de orden en $\partes(X)$ definida por:
  $A \relacion B \sisolosi A - B = \vacio$.

  ¿Cuántos conjuntos $A \en \partes(X)$ cumplen simultáneamente $\#A \geq 2$ y $A \relacion \set{1,2,3,4,5,6,7,8,9}$?
\end{enunciado}

Los ingredientes:
\begin{enumerate}[label=\purple{\faIcon{utensils}$_{(\arabic*)}$}]
  \item Para que se cumpla que $A \relacion B$ necesito que $A \en \partes(\set{1,2,3,4,5,6,7,8,9})$
  \item Además necesito que por lo menos tenga $\#A \geq 2$
\end{enumerate}

Por lo tanto es cuestión de ir agarrando elementos de $\partes(\set{1,2,3,4,5,6,7,8,9})$ teniendo en cuenta que esos
elementos tienen que tener \textit{un cardinal mayor o igual a $2$}. Por favor entender que los \textit{elementos del conjunto partes son conjuntos.}

\medskip

Esto es un trabajo para \simpleicon{burgerking}, digo el \textit{número combinatorio:}
$$
  \binom{9}{2} +
  \binom{9}{3} +
  \binom{9}{4} +
  \binom{9}{5} +
  \binom{9}{6} +
  \binom{9}{7} +
  \binom{9}{8} +
  \binom{9}{9} = \llamada1
$$
Y si querés averiguar cuanto \textit{verga} es eso ahí tenés un laburo para \simpleicon{mcdonalds}, digo el \textit{binomio de Newton:}
{\small
  $$
    \textstyle
    (x + y)^n = \sumatoria{k=0}{n}\binom{n}{k} x^n y ^{n-k}
    \Entonces{$x = y = 1$}[$n = 9$]
    2^9 =
    \sumatoria{k=0}{9}\binom{9}{k} =
    \ub{\binom{9}{0}}{=1} +
    \ub{\binom{9}{1}}{=9} +
    \ub{
      \binom{9}{2} +
      \binom{9}{3} +
      \binom{9}{4} +
      \binom{9}{5} +
      \binom{9}{6} +
      \binom{9}{7} +
      \binom{9}{8} +
      \binom{9}{9}
    }
    {
      \llamada1
    }
  $$
}
Despejando:
$$
  \cajaResultado{
    \llamada1 =
    \binom{9}{2} +
    \binom{9}{3} +
    \binom{9}{4} +
    \binom{9}{5} +
    \binom{9}{6} +
    \binom{9}{7} +
    \binom{9}{8} +
    \binom{9}{9} =
    2^9 - 10 = 502
  }
$$

Me voy a comer algo, \faIcon{handshake}.

\begin{aportes}
  \item \aporte{\dirRepo}{naD GarRaz \github}
\end{aportes}
